\documentclass{article}

\usepackage{anyfontsize}
\usepackage[letterpaper, margin=.5in]{geometry}
\usepackage{titlesec}

\titleformat{\section}{\fontsize{18}{18}\bfseries}{}{0em}{}

\newcommand{\cmdentry}[3]{
\section{#1} % Command title

\indent\indent
\textbf{Ex:} \textit{#2}\\ % Example command
\indent
#3 %Command explanation
\\
}

\title{Serial Commands for Countdown Display}
\author{Joshua Bays (UAH ETLC)}\date{}

\begin{document}\fontsize{14}{21}\selectfont
\maketitle
\tableofcontents\newpage

\cmdentry{setc}{setc 122520232015450000}{Set the clock to a given time in the MMDDYYYYhhmmss format.\\
\indent (Example above sets the clock to 12.25.2023 at 8:15:45 PM)}
\cmdentry{sett}{sett 00:55:05}{Set the timer to a specific time in the hh:mm:ss format.\\
\indent (Example above sets a timer for 55 minutes and 5 seconds)}
\cmdentry{swon}{swon}{Turn on the stopwatch. The stopwatch will display automatically.}
\cmdentry{swon}{swoff}{Turn off the stopwatch. The stopwatch will stop displaying automatically.}
\cmdentry{togc}{togc}{Toggle the clock from displaying.}
\cmdentry{togt}{togt}{Toggle the timer from displaying.}
\cmdentry{toga}{toga}{Toggle the alarm sound.}
\cmdentry{pzt}{pzt}{Pause/Resume the timer.}
\cmdentry{pzs}{pzs}{Pause/Resume the stopwatch}
\cmdentry{sec}{sec}{Toggle the seconds display on the clock}

\end{document}
